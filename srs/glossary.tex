\makenoidxglossaries
\newglossaryentry{product-backlog}{
  name={product Backlog},
  description={это упорядоченный и постоянно обновляемый список всего, что планируется сделать для создания и улучшения продукта. Этот артефакт Скрама является единственным источником работы для Скрам-команды. \Gls{product-owner} несет ответственность за Бэклог Продукта, включая его содержимое, доступность и упорядочение}
}
\newglossaryentry{product-owner}{
  name={product Owner},
  description={это одна из 3 зон ответственности в Скрам-команде. Владелец Продукта отвечает за максимизацию ценности продукта, получаемого в результате работы Скрам-команды. В его обязанности также входит курирование и приоритизация Бэклога Продукта. Около 50\% времени Владелец Продукта проводит с клиентами и заинтересованными лицами, остальные 50\% работает совместно с командой}
}
\newglossaryentry{kanban-board}{
  name={kanban доска},
  description={это способ визуализации списка задач. На странице Kanban-доски отображаются все задачи из \Gls{product-backlog} в форме карточек. При этом карточки можно перемещать между статусами.}
}
\newglossaryentry{task-attribute}{
  name={атрибут задачи},
  plural={атрибуты задачи},
  description={это  свойство, ассоциированные с задачей. В данной системе атрибутами задачи являются: название, описание, \gls{task-priority}}
}
\newglossaryentry{task-priority}{
  name={приоритет задачи},
  description={определяет, насколько задача важна для системы. Может быть одним из следующих значений: низкий, средний, высокий, блокер}
}

\newglossaryentry{latex}
{
    name=latex,
    description={Is a markup language specially suited for 
scientific documents}
}


\newglossaryentry{formula}
{
    name=formula,
    description={A mathematical expression}
}


