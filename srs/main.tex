\documentclass{article}
\usepackage[T2A]{fontenc}
\usepackage[utf8]{inputenc}
\usepackage{hyperref}
\usepackage[english,russian]{babel}
% For... glossary
\usepackage[toc,translate=babel]{glossaries}

% To set custom label for enumerate
\usepackage{enumitem}

\usepackage[left=2cm,right=2cm]{geometry}

\author{Егор Федоров \and Андрей Карабанов}
\title{Software Requirements for Zadachnik}

\makenoidxglossaries
\newglossaryentry{product-backlog}{
  name={product Backlog},
  description={это упорядоченный и постоянно обновляемый список всего, что планируется сделать для создания и улучшения продукта. Этот артефакт Скрама является единственным источником работы для Скрам-команды. \Gls{product-owner} несет ответственность за Бэклог Продукта, включая его содержимое, доступность и упорядочение}
}
\newglossaryentry{product-owner}{
  name={product Owner},
  description={это одна из 3 зон ответственности в Скрам-команде. Владелец Продукта отвечает за максимизацию ценности продукта, получаемого в результате работы Скрам-команды. В его обязанности также входит курирование и приоритизация Бэклога Продукта. Около 50\% времени Владелец Продукта проводит с клиентами и заинтересованными лицами, остальные 50\% работает совместно с командой}
}
\newglossaryentry{kanban-board}{
  name={kanban доска},
  description={это способ визуализации списка задач. На странице Kanban-доски отображаются все задачи из \Gls{product-backlog} в форме карточек. При этом карточки можно перемещать между статусами.}
}
\newglossaryentry{task-attribute}{
  name={атрибут задачи},
  plural={атрибуты задачи},
  description={это  свойство, ассоциированные с задачей. В данной системе атрибутами задачи являются: название, описание, \gls{task-priority}}
}
\newglossaryentry{task-priority}{
  name={приоритет задачи},
  description={определяет, насколько задача важна для системы. Может быть одним из следующих значений: низкий, средний, высокий, блокер}
}

\newglossaryentry{latex}
{
    name=latex,
    description={Is a markup language specially suited for 
scientific documents}
}


\newglossaryentry{formula}
{
    name=formula,
    description={A mathematical expression}
}




\begin{document}
\maketitle
\tableofcontents

\section{Описание информационной системы}
Информационная система Zadachnik -- система отслеживания багов и
управления проектами.
Основное предназначение системы -- организовывать управление задачами и багами
в IT проектах, реализующих Scrum.

Информационная система позволит решить следующие задачи:
\begin{itemize}
  \item Отслеживание задач в проекте
  \item Отслеживание связей между задачами
  \item Распределение задач между участникам
  \item Оценка времени на выполнение задач
  \item Оценка времени на выполнение всего проекта
  \item Сбор аналитики по времени выполнения задач
  \item Фасилитация и организация sprint planning, daily scrum, spring review и spring retro.
\end{itemize}

\subsection{Классы и характеристики пользователей}
Так как система построена для использования в Scrum-командах, то в ней
можно выделить следующие классы пользователей. Эти классы в основном выделены из
классических ролей в Scrum, в качестве источника был использован веб-сайт scrumtrek.ru
\begin{itemize}
\item Product Owner. Отвечает за максимизацию ценности продукта, получаемого в результате работы Scrum-команды.
В его обязанности также входит курирование и приоритизация бэклога продукта.
\item Scrum Master. Является лидером-слугой (Servant Leader) для Скрам-команды и для организации в целом.
Обучает команду устранять препятствия, является коучем команды и фасилитирует Мероприятия Скрама.
Фактически является владельцем процесса, ответственным за эффективную работу команды.
\item Developer. Разработчики -- это люди, работающие над Элементами Бэклога Спринта.
Они имеют все необходимые компетенции, чтобы каждый Спринт создавать работающий Инкремент Продукта.
\end{itemize}

\section{Функциональные требования}

В данной секции в формате user story описаны требования к системе.
Требования разбиты на подсекции в зависимости от класса пользователей,
от которых исходит это требование.
Если требование актуально для нескольких классов пользователей, оно
вынесено в общее.

\subsection{Общие требования}
\begin{enumerate}[label=\textbf{FR\arabic*}.]
  \item Я, как пользователь, хочу иметь возможность зарегистрироваться в системе с помощью логина и пароля
  \item Я, как пользователь, хочу указывать \glsdisp{task-attribute}{атрибуты} задачей, сторей и эпиков
  \item Я, как пользователь, хочу видеть \Gls{product-backlog} и бэклог спринта в виде \glsdisp{kanban-board}{Kanba-доски}
  \item Я, как пользователь, хочу создавать и обновлять задачи, чтобы управлять своими рабочими элементами
  \item Я, как пользователь, хочу менять статус задачи (например, "К выполнению", "В процессе", "Готово"), чтобы команда была в курсе прогресса по задаче
  \item Я, как пользователь, хочу взаимодействовать с членами команды через комментарии к задачам для уточнения требований и деталей реализации
\end{enumerate}

\subsection{Требования Product Owner}
\begin{enumerate}[label=\textbf{POR\arabic*}.]
  \item Я, как Product Owner, хочу иметь возможность создавать релиз продукта
  \item Создавать и приоритизировать бэклог продукта
\end{enumerate}

\subsection{Требования Scrum Master}
\begin{enumerate}[label=\textbf{SMR\arabic*}.]
  \item Я, как Scrum Master, хочу организовывать задачи в спринты
  \item Я, как Scrum Master, хочу просматривать Burndown Chart
  \item Я, как Scrum Master, хочу запланировать в системе
    Sprint Planning Meeting, Scrum Daily Meeting, Sprint Review, Sprint Retrospective
  \item Я, как Scrum Master, хочу назначать задачи разработчикам, чтобы равномерно распределить работу в зависимости от загрузки команды.
\end{enumerate}

\printnoidxglossaries
\end{document}

\section{Нефункциональные требования}
\begin{enumerate}[label=\textbf{NFR\arabic*}.]
  \item Система должна поддерживать отображения без нарушения работы функциональности
    и дизайна в браузерах Chrome 130+, Mozilla 120+
\end{enumerate}

\section{Прецеденты использования}
\subsection{Регистрация в сети}
\begin{center}
  \begin{tabular}{|l|p{9cm}|}
  \hline
  \textbf{Прецендент} & Регистрация в системе \\
  \hline
  \textbf{ID} & 1 \\
  \hline
  \textbf{Краткое описание} & Пользователь регистрируется в cистеме \\
  \hline
  \textbf{Главный актер} & Пользователь\\
  \hline
  \textbf{Второстепенные актеры} & нет \\
  \hline
  \textbf{Предусловия} &  \\
  \hline
  \textbf{Основной поток} & \begin{enumerate}
    \item Пользователь переходит на страницу регистрации
    \item Пользователь вводит свой логин.
    \item Пользователь вводит пароль
    \item Пользователь попадает на страницу настройки своего профиля для продолжения заполнения данных
  \end{enumerate} \\
  \hline
\end{tabular}
\end{center}
